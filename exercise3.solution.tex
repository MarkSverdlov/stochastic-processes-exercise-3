\documentclass{amsart}
\usepackage{graphicx} % Required for inserting images
\usepackage{amsthm, amsmath, amssymb, amsfonts} % general mathmode commands
\usepackage{bm} % for \bm
\usepackage{stmaryrd} % for \nnearrow
\usepackage{cancel} % for \cancel
\usepackage{breqn} % automatic line breaks in equations
\usepackage[margin=2.5cm]{geometry} % for page layout
\usepackage{cancel} % for \cancel
\usepackage{algorithm2e} % for algorithms
\usepackage{bbm} % for \mathbbm{1}
\usepackage{mathtools} % for \coloneq
\usepackage{hyperref}
\usepackage{nicefrac}
\usepackage{float}
\usepackage{biblatex}
\usepackage{lipsum}
\usepackage{placeins}
\usepackage{booktabs}
\usepackage{xcolor,colortbl}
\usepackage{makecell}
\RestyleAlgo{ruled}

% \new theorems definition
\theoremstyle{plain}
\newtheorem{theorem}{Theorem}
\newtheorem{corollary}[theorem]{Corollary}
\newtheorem{lemma}[theorem]{Lemma}
\newtheorem{proposition}[theorem]{Proposition}

\theoremstyle{definition}
\newtheorem{definition}[section]{Definition}

\theoremstyle{definition}
\newtheorem{example}[section]{Example}
\newtheorem{conjecture}[section]{Conjecture}

\theoremstyle{remark}
\newtheorem{remark}[section]{Remark}

% \newcommand
\DeclareMathOperator*{\argmax}{argmax}
\DeclareMathOperator{\err}{Err}
\DeclareMathOperator{\softmax}{softmax}
\newcommand{\diff}{\mathrm{d}}
\title{Stochastic Processes --- Exercise 3 --- Solution}
\author{Mark Sverdlov \\ 323454710}
\date{August 2025}

\begin{document}
\maketitle
    \section{Exercise 1}
        We consider the equation:
        \begin{gather}
            \label{eq:sde1}
            \diff X_{t} = 2X_{t}\diff B_{t} \\
            X_{0} = 3
        \end{gather}
        And we regard it as an integral equation with the Ito interpretation. We set \(g \left(x\right) = \log x\), and utilizing Ito's formula we get:
        \begin{equation}
            \label{eq:ito-formula}
            \log X_{t} = \log X_{0} + \int \frac{\diff X_{s}}{X_{s}} - \int \frac{1}{2X_{s}^{2}} \left(\diff X_{s}\right)^{2}
        \end{equation}
        Where \(\left(\diff X_{t}\right)^{2}\) is interpreted by Ito's formula convention for quadric variation, and the integral is the indefinite Ito's integral that passes through \(\left(0,\,0\right)\). By using~\eqref{eq:sde1} we get:
        \begin{equation}
            \label{eq:quadric-variation}
            \left(\diff X_{t}\right)^{2} = 4X_{t}^{2} \diff t
        \end{equation}
        and by putting~\eqref{eq:quadric-variation} into~\eqref{eq:ito-formula} we get:
        \begin{equation*}
           \log X_{t} = \log X_{0} + \int \frac{\diff X_{t}}{X_{t}} -2 \int \diff t
        \end{equation*}
        And therefore:
        \begin{equation}
            \label{eq:integration}
            \int \frac{\diff  X_{t}}{X_{t}} = \log X_{t} - \log X_{0} + 2t
        \end{equation}
        By rearranging~\eqref{eq:sde1} we get:
        \begin{equation*}
            \frac{\diff X_{t}}{X_{t}} = 2 \diff B_{t}
        \end{equation*}
        And by taking Ito's integral of both sides and putting in~\eqref{eq:integration} and the initial condition we get:
        \begin{equation*}
           \log X_{t} = \log 3 + 2 \left(B^{t}-t\right)
        \end{equation*}
        And thus:
        \begin{equation*}
            X_{t} = 3 e^{2\left(B_{t}-t\right)}
        \end{equation*}

        \section{Exercise 2}
        First, we set:
        \begin{equation*}
            X_{t} = \cos \left(t + B_{t}\right)
        \end{equation*}
        and
        \begin{equation*}
            g \left(x,\,y\right) = \cos \left(x + y\right)
        \end{equation*}
        From Ito's formula, when applied on \(g \left(t+B_{t}\right)\), we get:
        \begin{equation*}
            \diff  X_{t} = -\sin \left(t + B_{t}\right) \diff t -\sin \left(t + B_{t}\right) \diff  B_{t} - \frac{1}{2} \cos \left(t + B_{t}\right) \diff t
        \end{equation*}
        After rearranging and using the trigonometric identity \(\sin \left(x\right) = \sqrt{1-\cos^{2} \left(x\right)}\), we get:
        \begin{equation*}
            \diff X_{t} = -\left(\frac{1}{2}X_{t}+\sqrt{1-X_{t}^{2}}\right)\diff t - \sqrt{1-X_{t}^{2}}\diff B_{t}
        \end{equation*}
        Which together with the initial condition \(X_{0}=1\) has \(X_{t}\) as a solution.

        \section{Exercise 3}
        First, Suppose:
        \begin{gather}
            \label{eq:sde2}
            \diff  X_{t} = \left(\alpha-\mu X_{t}\right)\diff t + \sigma \diff B_{t} \\
            X_{0} = \frac{\alpha}{\mu}
        \end{gather}
        We set:
        \begin{equation}
            \label{eq:Y-id}
            Y_{t} = e^{\mu t}X_{t} - \frac{\alpha}{\mu}e^{\mu t}
        \end{equation}
        and by applying Ito's formula to \(g \left(x,\,y\right) = e^{\mu x}y - \frac{\alpha}{\mu}e^{\mu x}\), where \(x=t,\,y=X_{t}\) we get:
        \begin{equation*}
            \diff Y_{t} = \left(\mu e^{\mu t}X_{t}-\alpha e^{\mu t}\right) \diff t + e^{\mu t}\diff X_{t}
        \end{equation*}
        We then put in the original equation~\eqref{eq:sde2} and get:
        \begin{equation*}
            \diff Y_{t} = \sigma e^{\mu t}\diff B_{t}
        \end{equation*}
        By utilizing the initial condition that \(X_{0} = \frac{\alpha}{\mu}\) we get that \(Y_{0} = 0\), and using this we get the equation:
        \begin{equation}
            \label{eq:Y-char}
            Y_{t} = \int \sigma e^{\mu t}\diff B_{t}
        \end{equation}
        From~\eqref{eq:Y-char} we get two important facts about \(Y_{t}\). First, since it's an indefinite Ito's integral with respect to Brownian motion, we know that it's a martingale, and in particular that \(\mathrm{E}Y_{t}=\mathrm{E}Y_{0}=0\). Additionally, we may use Ito's isometry to get:
        \begin{equation*}
            \mathrm{Var}Y_{t} = \mathrm{E}Y_{t}^{2} = \mathrm{E}\int_{0}^{t} \left(\sigma e^{\mu t}\right)^{2} \diff t = \frac{\sigma^{2}}{2 \mu} \left(e^{2 \mu t} - 1\right)
        \end{equation*}
       Finally, by rearranging~\eqref{eq:Y-id} we get:
       \begin{equation*}
           X_{t} = \frac{\alpha}{\mu} + e^{-\mu t}Y_{t}
       \end{equation*}
       and by using the facts about \(Y_{t}\) we have established above, we readily arrive at the conclusion that:
       \begin{gather*}
           \mathrm{E}X_{t} = \frac{\alpha}{\mu} \\
           \mathrm{Var} X_{t} = \frac{\sigma^{2}}{2 \mu} \left(1 - e^{-2 \mu t}\right)
       \end{gather*}
       \end{document}
